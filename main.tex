\documentclass[usenames,dvipsnames]{article}
\usepackage[utf8]{inputenc}
\usepackage{amsmath, amsthm, amssymb, mathtools, xcolor}
\usepackage{graphicx}
\usepackage{caption}
\usepackage[shortlabels]{enumitem}

\usepackage[ruled,vlined,linesnumbered]{algorithm2e}
\usepackage{expl3}

\usepackage{tikz, pgfplots}
\usepgfplotslibrary{fillbetween}
\pgfplotsset{compat=1.14}
\usetikzlibrary{arrows,positioning,matrix,decorations.pathmorphing,calc} 

\tolerance 10000
\headheight 0in
\headsep 0in
\evensidemargin 0in
\oddsidemargin \evensidemargin
\textwidth 6.5in
\topmargin .25in
\textheight 8.7in

\theoremstyle{plain}
\newtheorem{theorem}{Theorem}[section]
\newtheorem{corollary}[theorem]{Corollary}
\newtheorem{proposition}[theorem]{Proposition}
\newtheorem{lemma}[theorem]{Lemma}
\newtheorem{conjecture}[theorem]{Conjecture} 
\newtheorem{ques}[theorem]{Question} 
\newtheorem{prob}[theorem]{Problem} 

\theoremstyle{definition}
\newtheorem{defn}[theorem]{Definition}
\newtheorem{example}[theorem]{Example}
\newtheorem{notation}[theorem]{Notation}
\newtheorem{condition}[theorem]{Condition}
\newtheorem{rmk}[theorem]{Remark}
\newtheorem{proc}[theorem]{Procedure}

\newcommand{\defword}[1]{\textbf{\textcolor{purple}{#1}}}
\newcommand{\red}[1]{\textbf{\textcolor{red}{#1}}}
\newcommand{\green}[1]{\textbf{\textcolor{ForestGreen}{#1}}}

\title{WBIO project notes}
\author{}
\date{June 2019}

\begin{document}

\maketitle

\section{Introduction}

\section{A mathematical model for R-loops}

\begin{prob}
Determine an abstract model to describe R-loops.
\end{prob}

\begin{defn}
Def of R-loop
\end{defn}

\subsection{Grammar}

Historical note: Knudsen-Hein first(?) used Grammar in the RNA setting as a model for RNA secondary structure: rules of Grammar produce some \emph{words}, and then words can be interpreted as a folding.  Can use training data to assign probabilities to each of the rules of the Grammar.
\begin{defn}
Give a definition for a grammar in general?
A \defword{grammar} $G$ is a 4-tuple $(S,V,T,R)$, where 
\begin{itemize}
    \item $V$ is a finite set of letters called \defword{non-terminals}, 
    \item $T$ is a finite set of letters called \defword{terminals},
    \item $R$ is a set of rules (which are maps $(V\cup T)^*\rightarrow (V\cup T)^*$ \footnote{Here, $A^*$ denotes the free monoid generated by $A$. In other words, $A^*$ is the set of words consisting of letters of $A$.}, and
    \item $S \in V$ is called the \defword{start symbol}.
\end{itemize}
We say that $S$ generates a word $W\in T^*$, denoted $S\Rightarrow^* W$ if there is a sequence of rules $R_1,R_2,\ldots,R_n$ such that $S \xrightarrow{R_1} W_1 \xrightarrow{R_2} W_2 \ldots \xrightarrow{R_n} W_n=W$ (here $W_1,\ldots,W_n\in (V\cup T)^*$ is a sequence of words
\footnote{We say that the grammar is \defword{non-ambiguous} if for each $W \in L(G)$, the sequence $W_1,\ldots,W_n$ is unique}. 
Then, the \defword{language} of the grammar, is $L(G):=\{ W \ |\ S\Rightarrow^* W\}.$ 

When $R\subseteq V\times (V\cup T)^*$, we say that the grammar $G$ is \defword{context-free}.
\end{defn}

Once you determine the language of the grammar, or even 
\begin{defn}
Our Grammar model for R-loops

The rules in the grammar
\begin{enumerate}
    \item \red{$S\rightarrow eS$}
    \item \red{$S\rightarrow rIR'S_1$}
    \item \red{$S_1\rightarrow e$}
    \item \red{$S_1\rightarrow e S_1$}
    \item \green{$I\rightarrow ii\ell I$}
    \item \green{$I\rightarrow iI$}
    \item \green{$I\rightarrow \varepsilon$}
\end{enumerate}

For a word $W\in G$ we define the length of $W$ to be the total number of $e$'s and $i$'s in the word. A generic word in the language will be of the form
$e^n r (ii\ell)^m i^k r' e^{n'}$
and hence the length will be $n + 2m + k + n'$.
\end{defn}

Biological interpretations
\begin{itemize}
    \item $S$ is the start symbol (the chunk of the R-loop \emph{before} the formation of the DNA-RNA hybrid)
    \item $S_1$ is the chunk of the R-loop \emph{after} the formation of the DNA-RNA hybrid
    \item $e$ is a half twist of DNA-DNA
    \item $I$ is the interior content (when the DNA-RNA duplex is hybridized(?) and the third DNA strand is looping around the duplex)
    \item $\ell$ is one pass of the free DNA strand over the DNA-RNA hybrid
    \item $i$ is the interior half-twist of DNA-RNA
    \item The length of the word will correspond to the number of nucleotides in the R-loop (which is 1500kb(?) nucleotides(?) \cite{}).
    \item $ii\ell$ represents the free DNA strand passing from back to front of the DNA-RNA duplex, or, the free DNA strand passing from front to back of the DNA-RNA duplex.
\end{itemize}

\begin{rmk}
Some biological limitations of our model:
\begin{itemize}
    \item How many $i$'s per $\ell$? If we have more supercoiling, then there will be more $i$'s per $\ell$.
\end{itemize}
\end{rmk}


\section{Shadows of Nucelic Acids}

\begin{ques}
Can we uniquely identify DNA from a picture (projection) of it?
\end{ques}

\begin{prob}
Give an algorithmic way to convert a picture/snapshot of DNA to a 4-valent graph.
\end{prob}


\end{document}
